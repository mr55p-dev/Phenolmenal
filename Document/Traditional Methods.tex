\subsection{Steam methane reformation}%
\label{sub:steam_methane_reformation}

The steam methane reformation reaction is an extremely inexpensive and widespread method for producing molecular hydrogen in a commercial setting.
The reaction follows the scheme\cite{Saxena2011}:
\begin{align}
	\ce{CH4(g) + H2O(l) &-> 3H2(g) + CO(g)} \label{ce:steam-reformation}\\
	\Delta H &= \SI{379}{\kilo\joule},\ \SI{1500}{K} \nonumber
.\end{align}
There is also a secondary mechanism for extracting hydrogen from the carbon monoxide by-product of this first stage reaction, which runs at a lower temperature and is called a water gas shift reaction\cite{Chen2008,Saxena2011}.
\begin{align}
	\ce{CO + H2O &-> CO2 + H2}\label{ce:water-shift} \\
	\Delta H &= \SI{-41.1}{\kilo\joule\per\mole},\ \SI{1200}{K} \nonumber
.\end{align}
This basic reaction scheme is applied to generate approximately 40\% of the global hydrogen demand\cite{SBN2020}; primarily due to using readily available reagents, it is a mature technique and it is inexpensive.

Although both reactions \eqref{ce:steam-reformation} and \eqref{ce:water-shift} can proceed without the use of a catalyst one is often employed for the water gas shift reaction \eqref{ce:water-shift} in order to increase the hydrogen yield.
Iron-chromium or copper-zinc compounds are often used dependant on if the reaction is a low- or high-temperature shift. EXPAND on this.

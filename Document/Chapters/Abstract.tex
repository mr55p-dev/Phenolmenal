The widespread use of fossil fuels is a major environmental issue, due to the release of greenhouse gases contributing to global warming.
Therefore, it is necessary to find an alternative, \ce{H2} could be a promising option, however problems persist in current methods for generation of \ce{H2}.
In this review we discuss the current methods of steam methane reformation and uncatalysed water electrolysis for the production of hydrogen and evaluate potential novel methods involving both methane and water as potential feedstocks.

The methane methods evaluated in this review are the thermal decomposition of methane using \ce{Ni} catalyst on a hydroxyapatite support and radical methane pyrolysis.
Radical methane pyrolysis displayed around a \SI{90}{\percent} decrease in enthalpy cost when compared to the traditional steam methane reformation, which is a significant energy saving.
The water electrolysis methods we looked at are solid oxide electrolytic cells and acidic/alkaline amphoteric electrolytic cells.
The latter achieved an energy consumption \SI{30}{\percent} less than conventional alkaline electrolysis.

These new methods have clear improvements over current methods in terms of energy cost, which makes the production of \ce{H2} using renewable energy a viable option.
This would transform the way we store and release energy.


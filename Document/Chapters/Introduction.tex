As the global energy demand continues to increase, a sustainable alternative to fossil fuels is desperately needed.
\ce{H2} potentially presents an innovative solution, but the energy costs for production are currently prohibitive.
An efficient synthesis could be a revolutionary step towards a more sustainable future, as the hydrogen fuel cell has many advantages over current energy sources; a hydrogen storage tank can be refilled much more quickly than a battery can be recharged, \cite{Offer2010} making it far more practical for use in transport.

In a fuel cell, electricity is produced by reducing oxygen at the cathode \eqref{eq:1.1} and hydrogen at the anode \eqref{eq:1.2}.
\begin{align}
	\ce{O2 + 4H+ + 4e- &-> 2H2O}	\label{eq:1.1}\\
	\ce{2H2 &-> 4H+ + 4e-} 		\label{eq:1.2}
\end{align}
This reaction leads to the production of only heat, water and electricity, \cite{6278114} which is a major advantage over systems which produce \ce{CO2}.
These reasons alone make investigating the sustainability of hydrogen production a worthwhile endeavour.

We will start by discussing the traditional methods of synthesis and where they fall short of the criteria required to produce hydrogen sustainably.
\cite{Saxena2011} Then we will look at new approaches in the areas of methane decomposition and water electrolysis reactions.
The two methane methods are methane decomposition using an \ce{Ni} catalyst with a hydroxyapatite support, and radical methane pyrolysis.
The two water electrolysis methods are electrolysis with an acid/alkaline amphoteric cell and solar powered steam electrolysis using a solid oxide electrolytic cell.

We will then compare these new methods with the traditional methods; evaluating these new methods and investigating their sustainability compared with their traditional counterparts.
The most promising method from each feedstock will be analysed further, concluding on if and how these new approaches can solve the problems preventing widespread adoption of hydrogen as a fuel.
We will focus on how these new reactions can improve the rate of reaction, allow for more mild reaction conditions, and reduce the cost of reactions, particularly in terms of energy.

We are concerned with the environmental impact of these processes, however discussing the implications of these at scale is beyond the scope of this review.
The practical problems of scaling up will not be discussed in this review however.
We will also not evaluate the uses of the hydrogen when generated, such as potential issues with and improvements to the hydrogen fuel cell itself.


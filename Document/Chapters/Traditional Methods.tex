\subsection{Steam methane reformation}%
\label{sub:steam_methane_reformation}

The steam methane reformation reaction is an extremely inexpensive and widespread method for producing molecular hydrogen in a commercial setting.
The reaction follows the scheme\cite{Saxena2011} in equation \eqref{ce:steam-reformation}, with $\Delta H = \SI{379}{\kilo\joule\per\mole}$ at \SI{1500}{\kelvin}.
\begin{align}
	\ce{CH4 + H2O &-> 3H2 + CO} \label{ce:steam-reformation}
.\end{align}
There is also a secondary mechanism for extracting hydrogen from the carbon monoxide by-product of this first stage reaction, which runs at a lower temperature and is called a water gas shift reaction\cite{Chen2008,Saxena2011}.
This reaction \eqref{ce:water-shift} has $\Delta H = \SI{-41.1}{\kilo\joule\per\mole}$ at \SI{1200}{\kelvin}.
\begin{align}
	\ce{CO + H2O &-> CO2 + H2}\label{ce:water-shift} 
.\end{align}
This basic reaction scheme is applied to generate approximately 40\% of the global hydrogen demand\cite{SBN2020}; primarily de to using readily available reagents, it is an inexpensive and mature technique.

Although both reactions \eqref{ce:steam-reformation} and \eqref{ce:water-shift} can proceed without the use of a catalyst one is often employed for the water gas shift reaction \eqref{ce:water-shift} in order to increase the hydrogen yield.
Iron-chromium or copper-zinc compounds are often used depending on if the reaction is a low- or high-temperature shift respectively.

Unfortunately this method doesn't produce hydrogen in a sustainable way; the production rate without the gas shift reaction is low, and produces one equivalent of \ce{CO} for every three of \ce{H2}.
When the gas shift reaction is used this toxic carbon monoxide is used to generate further \ce{H2}, as well as the greenhouse gas \ce{CO2}.
Placing additional strain on the planet by upscaling steam reformation processes will only lead to further damage to the planet, so this process is not useful when considering fuelling a hydrogen powered society.


\subsection{Uncatalysed electrolysis of water}%
\label{sub:uncatalysed_electrolysis_of_water}

Water electrolysis is another basic technique for extracting hydrogen, this time from water.
Water is an incredibly naturally abundant chemical making up a large fraction of the earth's surface.
When hydrogen is consumed by a fuel cell water is the sole product.
Water can be split into hydrogen and oxygen via an electrolytic cell; at the anode reaction \eqref{eq:UC_oxidation} occurs and at the cathode reaction \eqref{eq:UC_reduction} occurs.
\begin{align}
	\ce{2H2O &-> 4e- + 4H+ + O2} \label{eq:UC_oxidation} \\
	\ce{2H+ + 2e- &-> H2} \label{eq:UC_reduction}
.\end{align}
A common choice of electrolyte for these kinds of cells is \ce{KOH}; an alkaline cell is more efficient\cite{merrill2006}.
\ce{KOH} has the highest charge mobility and solubility in water of the group I and II hydroxides, which makes it the best choice in this case.

Catalytic design is important, as when uncatalysed the electrolytic approach to interconverting \ce{H2}, \ce{O2} and \ce{H2O} is not feasible.
In acidic or alkaline solution, the standard cell potential for an electrolytic cell splitting water is equal to the standard potential generated from recombination in a fuel cell, \SI{1.23}{\volt}\cite{Peng2020}.
In practice however, there are many barriers to this reaction which result in uncatalysed cells always requiring a potential difference greater than \SI{1.80}{\volt}.
There are various practical issues which cause the greatly decreased efficiency; resistance inside the cell due to work required to move protons, product gasses resulting in a bad connection between the surface of the electrode and the electrolyte, and resistances outside the cell such as the work required for electrons to be conducted through a wire.

Uncatalysed electrolysis is a foundational method on which the water splitting method is built - catalysts are often deposited on the anode and cathode to lower the reaction barrier and reduce the overpotential that afflicts this method.
There are many possible catalysts, almost all of which involve adsorption of protons onto the electrode surface followed by the addition of two supplied electrons, as given in equation \eqref{eq:UC_Hads}\cite{Leonard2012}.
\begin{align}
	\ce{H+ + e- &-> H_{ads}}\\
	\ce{2H_{ads} &-> H2}\\
	\ce{H+ + e- + H_{ads} &-> H2} \label{eq:UC_Hads}
.\end{align}
Solving the problems with overpotential using safe, effective and cheap catalysts is a promising area of study which could lead to electrolytically generating hydrogen on an industrial scale.

\subsection{Discussion of traditional methods}%
\label{sub:discussion_of_traditional_methods}
Both of these methods discussed offer the potential for large scale hydrogen generation; albeit at a cost.
For steam methane reformation (SMR) the price is paid with high emissions into the atmosphere, with the current \ce{CO2} emissions crisis this isn't a viable way to solve the problem of city-scale energy fixation into \ce{H2}.
For electrolysis, overpotential related to several inefficiencies with the method makes it unviable for similar reasons; the energy to create the \ce{H2} must come from somewhere.
Renewable energy sources currently account for \SI{27}{\percent} of the global energy supply as of 2020\cite{IEA2020} (projected), with much of the rest coming from fossil fuels.

Electrolytic methods currently have fewer environmental problems to solve, but tend to be inefficient which is a major problem when considering the application to energy storage.
There is much research into addressing this inefficiency; modified catalytic electrodes have taken major steps to increase the rate of gas evolution at lower potentials, and entirely different techniques utilising steam at high temperatures address scaling and efficiency concerns.

Steam methane reformation itself is a technique which has been improved in efficiency and yield over the years it has been used; membrane reactors which use modifications to the gas-shift reaction \eqref{ce:water-shift}\cite{Saxena2011} show promise.
Other approaches to using methane as a feedstock are discussed in this review, ones which can take the same material and obtain the same product without the need to collect \ce{CO2} back out of the atmosphere.


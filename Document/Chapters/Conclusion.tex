 \ce{H2O} can be obtained more easily than  \ce{CH4}, which requires extraction from natural gas or the decomposition of biomass.
 \ce{H2O} is also inert and therefore far safer than  \ce{CH4} which is flammable.
Additionally,  \ce{H2O} is far easier to transport because it is a liquid above \SI{0}{\celsius}.
\ce{CH4} is a gas with a boiling point of \SI{-161.5}{\celsius}\cite{Lemmon2017}, therefore to transport large quantities it must be stored at very low temperatures as liquified natural gas \cite{lngamerica}.

However, in some locations where  \ce{H2O} is sparse or there is an unreliable electricity supply,  \ce{CH4} is a good solution\cite{SBN2020}.
Therefore, innovation is required for methods using both feedstocks.

The recent developments in water electrolysis and methane decomposition covered by this review are very promising alternatives to the methods currently used commercially.
Each paper recognises the challenges facing the large-scale use of a new method and takes steps to address these issues.
A method using  \ce{CH4} that stands out is radical methane decomposition (RMD), which requires \SI{11}{\percent} of the enthalpy required for traditional steam methane reformation.
A method using  \ce{H2O} that stands out is solid oxide steam electrolysis (SOSE), which uses less energy than traditional electrolysis and can use solar thermal energy to provide a significant portion.

Neither of these techniques are perfect, for example the RMD produces carbon as a waste product which greatly limits the lifespan of the catalyst and SOSE has issues maintaining the steam supply.
However, with further refinement, these cutting-edge techniques have potential to replace the flawed methods currently favoured for the industrial production of  \ce{H2}.

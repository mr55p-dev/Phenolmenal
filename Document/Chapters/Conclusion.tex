Water electrolysis and methane decomposition are both very promising alternatives to traditional methods of hydrogen production.
Some of the recent developments covered by this review have taken steps to remedy the biggest issues facing their commercial use.
One of the methods which stands out is radical methane decomposition (RMD), which requires only \SI{10}{\percent} of the enthalpy required for traditional steam methane reformation.
The other is steam electrolysis using solid oxide electrolytic cells (SOEC), which uses significantly less energy than traditional electrolysis.

As water is more abundant than methane, it can be obtained far more easily than methane which requires extraction from natural gas or collection from the decomposition of biomass.
Water is inert and doesn’t require compression, meaning transport is much less dangerous and economical than the transport of CH4.
However, CH4 provides a very good solution to the hydrogen problem for locations with poor access to water or an unreliable electricity supply.1 


\ce{H2O} can be obtained more easily than  \ce{CH4}, which requires extraction from natural gas or the decomposition of biomass.
As \ce{H2O} is inert it is far safer than  \ce{CH4} which is flammable.
Additionally,  \ce{H2O} is far easier to transport because it is a liquid above \SI{0}{\celsius}.
\ce{CH4} is a gas with a boiling point of \SI{-161.5}{\celsius}\cite{Lemmon2017}, therefore to transport large quantities it must be stored at very low temperatures as liquefied natural gas \cite{lngamerica}.
However, in some locations where  \ce{H2O} is sparse or there is an unreliable electricity supply, \ce{CH4} is a good solution\cite{SBN2020}.
Innovation is required to develop methods using both feedstocks.

The recent developments in water electrolysis and methane decomposition covered by this review are very promising alternatives to the methods currently used commercially.
Each paper recognises the challenges facing the large-scale use of a new method and takes steps to address these issues.
Whilst RMP is a valuable method in parts of the world with unreliable electricity supply or poor access to water, the SOEC cells have the advantage of scalability and the potential use of solar energy as a clean power source. 
In addition, the reliance of RMP reactions on methane as a feedstock can make them unreliable, as methane is harder to obtain, transport and is more dangerous to handle.

Neither of these techniques are perfect, the RMP produces carbon as a waste product which greatly limits the lifespan of the catalyst and SOECs have issues maintaining the steam supply.
However, with further refinement, these cutting-edge techniques have potential to replace the flawed methods currently favoured for the industrial production of  \ce{H2}.


This research is looking to use the \ce{H2} from methane decomposition to upcycle waste gases, such as \ce{CO2}, from industry:
\begin{align}
	\ce{CO2 + 3H2 -> CH3OH + H2O}
.\end{align}

For this, a steady stream of \ce{H2} gas is needed, which simply cannot be provided by water electrolysis due to the reasons outlined previously.
Methane pyrolysis has an enthalpy of reaction ($\Delta_RH$) of \SI{+37.5}{\kilo\joule\per\mole}, while the electrolysis of water has a $\Delta_RH$ of \SI{+268}{\kilo\joule\per\mole}, making methane decomposition more thermodynamically favourable than the electrolysis of water.
Methane pyrolysis occurs via a free radical dissociation reaction, involving one of a selection of catalysts available.
The initial step is the chemisorption of methane onto the catalytic surface.


This reaction mechanism gives rise to no waste gases such as \ce{CO2} but does has the issue of elemental carbon production which can lead to the degradation of the catalyst as carbon is deposited upon the catalyst surface (catalytic poisoning).

Methane pyrolysis occurs via a free radical dissociation reaction, involving one of a selection of catalysts available.
The initial step is the chemisorption of methane onto the catalytic surface.

Equations 3-81

This reaction mechanism gives rise to no waste gases such as \ce{CO2} but does has the issue of elemental carbon production which can lead to the degradation of the catalyst as carbon is deposited upon the catalyst surface (catalytic poisoning).


Iron can also be used as a catalyst, its catalytic activity is lower than that of nickel and cobalt, however it is more resistant catalytic deactivation.
This is because carbon has higher rate of diffusion in iron, meaning it does not deposit itself upon the active site of the catalyst.
It also has the advantage of being able to operate at high temperatures 700 to 1000°C.
It is also cheaper and less toxic than nickel and cobalt catalysts.

Both supported and non-supported iron catalysts have been tested such as: [Fe(CO)5], [Fe(cp)2].
However, these types of clusters can give result in unwanted gases meaning the \ce{H2} has to be separated from the gaseous mixture.
A catalyst was strong support increases the carbon dispersion and reducibility of the metal along with preventing sintering of metal particles.
However, an excessively strong support may hamper the reducibility of metal oxides.
A supported catalyst has a better performance as it balances the carbon dispersion resulting in a longer lasting catalyst while maintaining the reducibility of the metal species.

Ni-Pd or Ni-Cu can be used as a catalyst promoter; they have high lattice constants meaning they have a high capacity for waste carbon build up meaning they are longer living catalysts.
While these alloys are longer lasting, they have a lower rate of reaction.
The palladium or copper in the alloy serve as dopants, helping to initiate the hydrogen spill over effect.

Metal catalysts have reduced performance as the reaction progresses as the active sites becomes encapsulated by elemental carbon, therefore carbon-based catalysts should be considered.
These catalysts need higher operating temperatures have lower catalytic activity but are lower cost, have a high resistance to temperature and have a high stability, do not deposit toxic materials into the carbon by-product, tolerate impurities in the reaction such as sulphur and produce zero \ce{CO2} emissions as the catalyst doesn’t need to be regenerated.

There are three types of carbon catalyst; highly ordered, less ordered, and disordered.
Disordered carbon catalysts have free valence sites or coordination sites these are known as high energy sites (HES).
The more HES’s there are the faster the rate, as these are thought to initiate the mechanism.
More oxygenated catalysts have a higher activity but release \ce{COx}.

This is because as \ce{COx} is produced new active sites are created on the surface of the catalyst.
The reaction mechanisms for many of these carbon catalysts are not known.

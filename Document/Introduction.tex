As the global energy demand continues to increase, a sustainable alternative to fossil fuels is desperately needed.
Hydrogen gas potentially presents an innovative solution, but the energy costs for production are currently prohibitive.
An efficient synthesis could be a revolutionary step towards a more sustainable future as the hydrogen fuel cell has many advantages over current energy sources; a hydrogen storage tank can be refilled much more quickly than a battery can be recharged\cite{Offer2010}, making it far more practical for use in transport.

In a fuel cell, electricity is produced by reducing oxygen at the cathode \eqref{eq:1.1} and hydrogen at the anode \eqref{eq:1.2}.
\begin{align}
	\ce{O_2 + 4H+ + 4e- &-> 2H2O}	\label{eq:1.1}\\
	\ce{2H2 &-> 4H+ + 4e-} 		\label{eq:1.2}
\end{align}

This reaction leads to the production of heat, water and an electrical potential\cite{6278114}, which is a major advantage over systems which produce \ce{CO2}.
These reasons alone make investigating the sustainability of hydrogen production a worthwhile endeavour.

We will start by discussing the traditional methods of synthesis and where tey fall short by taking into account the twelve principles of green Chemistry\cite{Saxena2011}.
Then we will look at new approaches in the areas of methane decomposition and water splitting reactions, and how these compare to the traditional synthetic techniques.
The most promising method from each feedstock will be analysed further, concluding on if and how these new approaches can solve the problems preventing widespread adoption of hydrogen as a fuel.

We will focus on how the use of metal catalysts and how they can improve the rate of reaction, allow for more mild reaction conditions, and reduce the cost of reactions.
While we are concerned with the environmental impact of these processes, this review will not discuss the sourcing of the feedstock for each reaction, but will assume the methane being used in these reactions comes from carbon neutral sources such as biomass\cite{Probstein1982} and not from petrochemical routes.
We also realise that in order for a process to be viable on an industrial scale it must be possible to scale up the reaction, however the practical implications of scaling up will not be discussed in this review.
